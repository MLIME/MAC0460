\documentclass[10pt]{beamer}
\usetheme{metropolis}
% all imports
\usepackage[utf8]{inputenc}
\usepackage[T1]{fontenc}
\usepackage{lmodern}
\usepackage{appendixnumberbeamer}
\usepackage{hyperref}
\usepackage{booktabs}
\usepackage{bm}
\usepackage[scale=2]{ccicons}
\usepackage[outputdir=build]{minted}
\usepackage{pgfplots}
\usepackage{array,colortbl,xcolor}
\usepgfplotslibrary{dateplot}
\usepackage{setspace}
\usepackage{etoolbox}
\usepackage{xspace}
\usepackage{tikz}
\usetikzlibrary{shapes,arrows,positioning,fit,backgrounds}
\usepackage{tkz-euclide}
\usepackage{soul}
\usepackage{ragged2e}


\AtBeginEnvironment{quote}{\singlespacing}

% new commands
\newcommand{\themename}{\textbf{\textsc{metropolis}}\xspace}
\newcommand{\vect}[1]{\bm{#1}}
\newcommand{\myprime}[1]{{#1}^{\prime}}
\newcommand{\grad}[2]{\nabla_{#1} {#2}}
\newcommand{\dotp}[2]{{#1}^{\top}{#2}}
\newcommand{\dotpPright}[2]{{#1}^{\top}\left({#2}\right)}
\newcommand{\outerp}[2]{\left({#1}\right){#2}^{\top}}
\newcommand{\Jacobian}[2]{\frac{\partial #1}{\partial #2}}
\newcommand{\Vocab}{\mathbb{V}}
\DeclareMathOperator*{\argmin}{arg\,min}

% Quote with author reference at the end
\let\oldquote\quote
\let\endoldquote\endquote
\renewenvironment{quote}[2][]
  {\if\relax\detokenize{#1}\relax
     \def\quoteauthor{#2}%
   \else
     \def\quoteauthor{#2~---~#1}%
   \fi
   \oldquote}
  {\par\nobreak\smallskip\hfill(\quoteauthor)%
   \endoldquote\addvspace{\bigskipamount}}
%-----------------------------------------   


% definitions
\definecolor{blue}{RGB}{159, 192, 176}
\definecolor{green}{RGB}{160, 227, 127}
\definecolor{orange}{RGB}{243, 188, 125}
\definecolor{red}{RGB}{253, 123, 84}
\definecolor{nephritis}{RGB}{39, 174, 96}
\definecolor{emerald}{RGB}{46, 204, 113}
\definecolor{turquoise}{RGB}{39, 174, 96}
\definecolor{green-sea}{RGB}{22, 160, 133}
% Tikzstyles for Computation Graphs

% nodes
\tikzstyle{noop} = [circle, draw=none, fill=red, minimum size = 10pt]
\tikzstyle{op} = [circle, draw=red, line width=1.5pt, fill=red!70, text=black, text centered, font=\bf \normalsize, minimum size = 25pt]
\tikzstyle{state} = [circle, draw=blue, line width=1.5pt, fill=blue!70, text=black, text centered, font=\bf \normalsize, minimum size = 25pt]
% \tikzstyle{gradient} = [circle, draw=green, line width=1.5pt, fill=green!60, text=black, text centered, font=\bf \normalsize, minimum size = 25pt]
\tikzstyle{gradient} = [circle, draw=nephritis, line width=1.5pt, fill=nephritis!60, text=black, text centered, font=\bf \normalsize, minimum size = 25pt]
\tikzstyle{textonly} = [draw=none, fill=none, text centered, font=\bf \normalsize]
\tikzstyle{boxtextonly} = [draw=none, fill=none, align=center, font=\bf \normalsize]

% edges
% \tikzstyle{tedge}  = [draw, thick, >=stealth, ->]
\tikzstyle{tedge}  = [draw, thick, >=latex, ->]

% namedscope
\tikzstyle{namedscope} = [circle, draw=orange, line width=1.5pt, fill=orange!60, align=center, inner sep=0pt]

% \tikzstyle{container} = [draw=none, rectangle, dotted, inner ysep=1.5em]
% \tikzstyle{novertex} = [draw=none, fill=none, text centered]
% \tikzstyle{predicate} = [ellipse, draw, thick, text centered, rounded corners, minimum size=30pt]
% \tikzstyle{aux} = [rectangle, draw, thick, text centered, rounded corners, minimum size=30pt]
% \tikzstyle{ledge}  = [draw, dashed, thick, >=stealth, ->]
% \tikzstyle{pedge}  = [draw, thick, >=stealth, ->]



\title{MAC0460 - Computational Learning \\ Lecture 1} 
\date{\today}

\author{
  Nina S. T. Hirata\\
  \url{https://www.ime.usp.br/~nina/}
  \vspace{0.4 cm}
  \and\\ 
  Felipe Salvatore\\
  \url{https://felipessalvatore.github.io/}\vspace{0.4 cm}
}

\institute{\textbf{IME-USP}: Institute of Mathematics and Statistics, University of São Paulo}

\titlegraphic{
            \hspace{8.3cm}
            \includegraphics[scale=0.5]{images/logo.png}
}



\begin{document}
\nocite{DeepLearningbook}

\maketitle

\section{Revisão: regressão logística}

\begin{frame}{O problema de classificação}

\begin{itemize}
\item [] Em vários casos a função desconhecida $f:\mathbb{R}^{d} \rightarrow \mathbb{R}$ que queremos aproximar é uma \textbf{distribuição de probabilidade}.
\vspace{0.3cm}
\item[] Temos um vetor $\vect{x}$ e queremos saber a qual das classes $k_1, \dots, k_n$ ele pertence. Um modo de formular esse problema como um problema de apreendizado supervisionado é coletar um conjunto de dados $(\vect{x}_{1}, y_{1}), \dots ,(\vect{x}_{N}, y_{N})$ onde $y_i \in \{k_1, \dots, k_n\}$ e tentar estimar $p(y  | \vect{x})$ por meio de uma família de modelos $p(y | \vect{x}; \vect{\theta})$.
\end{itemize}
\end{frame}


\begin{frame}{Classificação com duas classes}
Quando $y$ é uma variável binária definimos o modelo $p(y | \vect{x}; \vect{\theta})$ do seguinte modo:
\Large{
\begin{align*}
\hat{y} &= p(y=1| \vect{x}; \vect{\theta})\\
&= h(\vect{x}; \vect{\theta}) \\
&= \sigma(z)\\
\end{align*}
}
em que 
\begin{equation*}
z = \dotp{\vect{w}}{\vect{x}} + b
\end{equation*}

\end{frame}

\begin{frame}[fragile]{Revisão: função sigmoide}
\begin{figure}[ht!]
\centering

\scalebox{1.0}{
\begin{tikzpicture}
    \begin{axis}%
    [
        grid=major,     
        xmin=-6,
        xmax=6,
        axis x line=bottom,
        ytick={0,.5,1},
        ymax=1,
        axis y line=middle,
    ]
        \addplot%
        [	orange!180,
        	ultra thick,
%             blue,%
            mark=none,
            samples=100,
            domain=-6:6,
        ]
        (x,{1/(1+exp(-x))});
    \end{axis}
\node[textonly] (sigmoid) at (8.75,2.8) {{\Large$\sigma(x) = \frac{1}{1 + e^{-x}}$}};
\end{tikzpicture}
} % scalebox
\end{figure}
\end{frame}

\begin{frame}{Classificação}
\input{TikzFiles/DFNclassification2}
\end{frame}

\begin{frame}{Classificação para várias classes}
E quando $y$ é uma variável com $n$ valores definimos $p(y | \vect{x}; \vect{\theta})$ do seguinte modo:
\Large{
\begin{align*}
\hat{\vect{y}} &= p(y| \vect{x}; \vect{\theta})\\
&= h(\vect{x}; \vect{\theta}) \\
&= softmax(\vect{z})\\
\end{align*}
}
em que 
\begin{equation*}
\vect{z} = \vect{W}\vect{x} + \vect{b}
\end{equation*}

\end{frame}

\begin{frame}[fragile]{Revisão: função softmax}
\begin{figure}[ht!]
\centering

\scalebox{1.3}{
\begin{tikzpicture}[auto]

% operations =============================

% nodes
\node[textonly] (logits) {$\begin{bmatrix}3.82\\5.35\\1.44\\-1.26\\2.71 \\1.98\end{bmatrix}$};
\node[textonly, right=60pt of logits] (softmax) {$\begin{bmatrix}0.16115195\\0.74422819\\0.01491471\\0.00100235\\0.05310907 \\0.02559374\end{bmatrix}$};
\node[textonly, below=15pt of logits] (inv1) {};
\node[textonly, right=10pt of inv1] (softmax_eq) {{\Large$softmax(\vect{x})_i = \frac{e^{\vect{x}_i}}{\sum_j e^{\vect{x}_j}}$}};



% edges
\path[tedge] (logits) edge node[above=1pt] {{\Large softmax}} (softmax);
\end{tikzpicture}
} % scalebox
\end{figure}

\end{frame}

\begin{frame}{Classificação}
\begin{figure}[ht!]
\centering

\scalebox{1.3}{
\begin{tikzpicture}[auto]

% operations =============================

% nodes
\node[textonly] (vectorx) {$\begin{bmatrix}0.34\\ \vdots \\0.06\end{bmatrix}$};
\node[textonly, above=1pt of vectorx] (x) {$\vect{x}$};
\node[textonly, below=1pt of vectorx] (dimension1) {{\small$d\times 1$}};
\node[op, right=30pt of vectorx] (model) {$h(\vect{x}; \vect{\theta})$};
\node[textonly, right=30pt of model] (vectoryhat) {$\begin{bmatrix}p(y=1| \vect{x};\vect{\theta})\\ \vdots \\p(y=n| \vect{x};\vect{\theta})\end{bmatrix}$};
\node[textonly, above=1pt of vectoryhat] (yhat) {$\hat{\vect{y}}$};
\node[textonly, below=1pt of vectoryhat] (dimension2) {{\small$n\times 1$}};



% edges
\path[tedge] (vectorx) -- (model);
\path[tedge] (model) -- (vectoryhat);


\end{tikzpicture}
} % scalebox
\end{figure}

\end{frame}


\begin{frame}{Princípio da máxima verossimilhança}
Os parâmetros $\vect{\theta}$ vão ser adaptados de modo que  $p(y| \vect{x};\vect{\theta})$ seja a distribuição mais adequada para os dados
\begin{equation*}
(\vect{x}^{(1)},y^{(1)}), \dots, (\vect{x}^{(N)},y^{(N)})
\end{equation*}
\end{frame}

\begin{frame}{Classificação}
A função que queremos maximizar é
\Large{
\begin{align*}
\mathcal{L}(\vect{\theta}) &= \E_{\vect{x},y \sim p_{data}} \log p(y| \vect{x}; \vect{\theta})\\
&= \frac{1}{N}\sum_{i=1}^{N}\log p(y^{(i)}| \vect{x}^{(i)}; \vect{\theta})\\
\end{align*}
}
\end{frame}

\begin{frame}{Revisão: entropia}
\begin{figure}[ht!]
\centering

\scalebox{1.3}{
\begin{tikzpicture}[auto]

% operations =============================

% nodes
\node[textonly] (pprob) {$\begin{bmatrix}0.8\\0.2\end{bmatrix}$};
\node[textonly, right=40pt of pprob] (qprob) {$\begin{bmatrix}0.5\\0.5\end{bmatrix}$};
\node[textonly, above=1pt of pprob] (p) {$\vect{p}$};
\node[textonly, above=1pt of qprob] (q) {$\vect{q}$};


\node[textonly, below=20pt of pprob] (Hp) {$H(\vect{p}) = 0.72$};
\node[textonly, below=20pt of qprob] (Hq) {$H(\vect{q}) = 1$};
\node[textonly, below=30pt of Hp] (inv1) {};
\node[textonly,  right=-40pt of inv1] (Hquation) {{\Large$H(\vect{p}) = \sum_{i} \vect{p}_i\log\frac{1}{\vect{p}_i}$}};


\end{tikzpicture}
} % scalebox
\end{figure}

\end{frame}

\begin{frame}{Revisão: entropia}
\begin{figure}[ht!]
\centering

\scalebox{1.0}{
\begin{tikzpicture}
    \begin{axis}%
    [
        grid=major,     
        xmin=0,
        xmax=1,
        axis x line=bottom,
        ytick={0,.5,1},
        ymax=1.1,
        axis y line=middle,
		xlabel= $p$,
  		ylabel= $H(\vect{p})$,
    ]
        \addplot%
        [	orange!180,
        	ultra thick,
%             blue,%
            mark=none,
            samples=200,
            domain=0.0001:0.9999,
        ]
        (x,{(x*log2(1/x)) + ((1-x)*log2(1/(1-x)))});
    \end{axis}
\node[textonly] (pprob) at (8.75,2.8) {{\Large$\begin{bmatrix}p\\1-p\end{bmatrix}$}};
\node[textonly, above=1pt of pprob] (p) {{\Large$\vect{p}$}};
\end{tikzpicture}
} % scalebox
\end{figure}
\end{frame}

\begin{frame}{Revisão: divergência Kullback-Leibler}
\begin{figure}[ht!]
\centering

\scalebox{1.3}{
\begin{tikzpicture}[auto]

% operations =============================

% nodes
\node[textonly] (p1prob) {$\begin{bmatrix}0.8\\0.2\end{bmatrix}$};
\node[textonly, right=30pt of p1prob] (q1prob) {$\begin{bmatrix}0.5\\0.5\end{bmatrix}$};
\node[textonly, above=1pt of p1prob] (p1) {$\vect{p}$};
\node[textonly, above=1pt of q1prob] (q1) {$\vect{q}$};
\node[textonly, right=20pt of q1prob] (p2prob) {$\begin{bmatrix}0.8\\0.2\end{bmatrix}$};
\node[textonly, right=30pt of p2prob] (q2prob) {$\begin{bmatrix}0.88\\0.12\end{bmatrix}$};
\node[textonly, above=1pt of p2prob] (p2) {$\vect{p}^{\prime}$};
\node[textonly, above=1pt of q2prob] (q2) {$\vect{q}^{\prime}$};


\node[textonly, below right=20pt and -15pt of p1prob] (Dkl1) {$D_{KL}(\vect{p}||\vect{q}) = 0.28$};
\node[textonly, below right=20pt and -15pt of p2prob] (Dkl2) {$D_{KL}(\vect{p}^{\prime}||\vect{q}^{\prime}) = 0.04$};
\node[textonly, below=20pt of Dkl1] (inv1) {};
\node[textonly,  right=-40pt of inv1] (Dklequation) {{\Large$D_{KL}(\vect{p}||\vect{q}) = \sum_{i} \vect{p}_i\log\frac{\vect{p}_i}{\vect{q}_i}$}};



% edges
\draw[orange!120, line width=1mm]  (Dkl1) to [out=150,in=-90] (p1prob);
\draw[orange!120, line width=1mm] (Dkl1) to [out=150,in=-100] (q1prob);

\draw[orange!120, line width=1mm]  (Dkl2) to [out=150,in=-90] (p2prob);
\draw[orange!120, line width=1mm] (Dkl2) to [out=150,in=-100] (q2prob);

\end{tikzpicture}
} % scalebox
\end{figure}

\end{frame}

\begin{frame}{Revisão: entropia cruzada}
\Large{
\begin{align*}
CE(\vect{p},\vect{q}) &= H(\vect{p}) + D_{KL}(\vect{p}||\vect{q})\\
\vspace{0.2cm}
&= -\sum_{i}\vect{p}_{i}\log(\vect{q}_{i})
\end{align*}
}
\vspace{0.2cm}
\begin{equation*}
\argmin_{\vect{q}} CE(\vect{p},\vect{q}) =  \argmin_{\vect{q}} D_{KL}(\vect{p},\vect{q})
\end{equation*}
\end{frame}

\begin{frame}[fragile]{Entropia cruzada e verossimilhança}

Assumindo que $\vect{y}$ é one-hot temos que: 

\Large{
\begin{align*}
L(\vect{y}^{(i)}, {\hat{\vect{y}}}^{(i)}) &= CE(\vect{y}^{(i)}, {\hat{\vect{y}}}^{(i)})\\
&= -\sum_{k=1}^{n} \vect{y}^{(i)}_{k}\log p(y=k| \vect{x}^{(i)}; \vect{\theta})\\
&= - \log p(y^{(i)}| \vect{x}^{(i)}; \vect{\theta})\\
\end{align*}
}
\end{frame}

\begin{frame}{Entropia cruzada e verossimilhança}
E a função que queremos minimizar é
\Large{
\begin{align*}
J(\vect{\theta}) &= \frac{1}{N}\sum_{i=1}^{N} L(\vect{y}^{(i)}, {\hat{\vect{y}}}^{(i)})\\
&= - \frac{1}{N}\sum_{i=1}^{N}\log p(y^{(i)}| \vect{x}^{(i)}; \vect{\theta})\\
&= - \mathcal{L}(\vect{\theta})
\end{align*}

\vspace{0.2cm}
\begin{equation*}
\argmax_{\vect{\theta}} \mathcal{L}(\vect{\theta}) =  \argmin_{\vect{\theta}} J(\vect{\theta})
\end{equation*}
}
\end{frame}

\begin{frame}{Treinando um modelo}
\Large{
\begin{itemize}
\item $\hat{\vect{y}} = f(\vect{x}; \vect{\theta})$ 
\item $J(\vect{\theta}) =  \frac{1}{m}\sum_{i=1}^{m} L(\vect{y}^{(i)}, {\hat{\vect{y}}}^{(i)})$ 
\item \text{algum algoritmo de otimização (e.g., \textbf{SGD})}:
\begin{equation*}
\vect{\theta}^{novo}  \leftarrow \vect{\theta}^{velho} - \eta \grad{\vect{\theta}}{J(\vect{\theta})}
\end{equation*}
\vspace{0.3cm}
\end{itemize}
}

Vamos ver como computar $\grad{\vect{\theta}}{J(\vect{\theta})}$ de modo eficiente para uma função arbitrária $J$.

\end{frame}

\section{Grafo de computação (caso escalar)}

\begin{frame}{Grafo de computação}

Considere os seguintes  conjuntos de funções:
\Large{
\begin{itemize}
\item $OP_1 = \{ \lambda x. -x, \lambda x. x^2,  \lambda x. e^x, \lambda x. log(x), \lambda x. x \}$
\item $OP_2 = \{ \lambda xy. x + y, \lambda  xy. x * y, \lambda xy. \frac{x}{y} \}$
\item $OP = OP_1 \cup OP_2$
\end{itemize}
}
\end{frame}

\begin{frame}{Grafo de computação}
Um grafo de computação definido em $OP$ $\mathcal{G} = (\mathcal{V}, \mathcal{E}_1, \mathcal{E}_2)$ é um grafo acíclico dirigido (DAG) tal que cada elemento $u \in \mathcal{V}$ indica uma variável, se $(x,y) \in \mathcal{E}_1$ então $f(x)=y$ onde $f \in OP_1 \cup \{g(x,\alpha) | \alpha \in \mathbb{R} , g \in OP_2\}$, e se $(x,y) \in \mathcal{E}_2$ então $f(x)=y$ onde $f \in \{g(\alpha, x) | \alpha \in \mathbb{R} , g \in OP_2\}$.

\vspace{0.3cm}
\begin{itemize}
\item $Pa(x) = \{y \in \mathcal{V} | (y,x) \in \mathcal{E}_1 \cup \mathcal{E}_2 \}$.
\item $S(x) = \{y \in \mathcal{V} | (x,y) \in \mathcal{E}_1 \cup \mathcal{E}_2 \}$.
\end{itemize} 

\end{frame}


\begin{frame}{Grafo de computação}

\input{TikzFiles/simple_example0}
\Large{
\begin{itemize}
\item $y  = x^2$
\item $u = e^{y}$
\end{itemize}
}
\end{frame}

\begin{frame}{Grafo de computação}

\begin{figure}[ht!]
\centering

\scalebox{1.0}{
\begin{tikzpicture}[auto]

% operations =============================
\node[op] (z) {$z$};
\node[op, above left=50pt of z] (x) {$x$};
\node[op, below left=40pt of z] (y) {$y$};
\node[op, right=40pt of z] (u) {$u$};


% edges =============================
\path[tedge] (x) -- (z);
\path[tedge] (y) -- (z);
\path[tedge] (z) -- (u);
\end{tikzpicture}
} % scalebox
\end{figure}

\Large{
\begin{itemize}
\item $z  = x + y$
\item $u = \log(z)$
\end{itemize}
}

\end{frame}

\begin{frame}{Grafo de computação}
\Large{
Queremos representar uma função $L$ por um grafo definido em $OP$ pois as derivadas parciais das funções de $OP$ são simples de calcular. E com a \alert{a regra da cadeia} podemos combinar as derivadas das funções locais para obter a derivada parcial de $L$ com respeito a quaisquer parâmetros.
}
\end{frame}

\begin{frame}{Grafo de computação}
\Large{
Como todas as funções em $OP$ são diferenciáveis, podemos extender $\mathcal{G}$ em $\mathcal{G}^{\prime}$ adicionando todas as derivadas parciais dos filhos em relação aos pais junto com as respectivas dependências. 
}
\end{frame}


\begin{frame}{Extendendo o grafo de operações básicas: soma}
\begin{figure}[ht!]
\centering

\scalebox{1.2}{
\begin{tikzpicture}[auto]

% operations =============================
\node[op] (times) {$+$};
\node[op, above left=20pt of times] (a) {$a$};
\node[op, below left=20pt of times] (b) {$b$};
\node[gradient, above=15pt of a] (da) {$\frac{\partial f}{\partial a}$};
\node[gradient, below=15pt of b] (db) {$\frac{\partial f}{\partial b}$};
\node[textonly, right=0.1pt of da] {$=1$};
\node[textonly, right=0.1pt of db] {$=1$};
\node[textonly, right=0.1pt of times] {$=f(a,b) = a+b$};

% edges
\path[tedge] (a) -- (times);
\path[tedge] (b) -- (times);


\end{tikzpicture}
} % scalebox
\end{figure}

\end{frame}

\begin{frame}{Extendendo o grafo de operações básicas: multiplicação}
\begin{figure}[ht!]
\centering

\scalebox{1.2}{
\begin{tikzpicture}[auto]

% operations =============================
\node[op] (times) {$\times$};
\node[op, above left=20pt of times] (a) {$a$};
\node[op, below left=20pt of times] (b) {$b$};
\node[gradient, above left=15pt and 20pt of a] (da) {$\frac{\partial f}{\partial a}$};
\node[gradient, below left=15pt and 20pt of b] (db) {$\frac{\partial f}{\partial b}$};
\node[textonly, right=0.1pt of da] {$=b$};
\node[textonly, right=0.1pt of db] {$=a$};
\node[textonly, right=0.1pt of times] {$=f(a,b) =a*b$};

% edges
\path[tedge] (a) -- (times);
\path[tedge] (b) -- (times);
\path[tedge] (a) -- (db);
\path[tedge] (b) -- (da);


\end{tikzpicture}
} % scalebox
\end{figure}

\end{frame}

\begin{frame}{Extendendo o grafo de operações básicas: divisão}
\begin{figure}[ht!]
\centering

\scalebox{1.2}{
\begin{tikzpicture}[auto]

% operations =============================
\node[op] (times) {$div$};
\node[op, above left=20pt of times] (a) {$a$};
\node[op, below left=20pt of times] (b) {$b$};
\node[gradient, above left=15pt and 20pt of a] (da) {$\frac{\partial f}{\partial a}$};
\node[gradient, below left=15pt and 20pt of b] (db) {$\frac{\partial f}{\partial b}$};
\node[textonly, right=0.1pt of da] {$=\frac{1}{b}$};
\node[textonly, right=0.1pt of db] {$=\frac{-a}{b^{2}}$};
\node[textonly, right=0.1pt of times] {$=f(a,b) =\frac{a}{b}$};

% edges
\path[tedge] (a) -- (times);
\path[tedge] (b) -- (times);
\path[tedge] (b) -- (da);
\path[tedge] (a) -- (db);
\path[tedge] (b) -- (db);


\end{tikzpicture}
} % scalebox
\end{figure}

\end{frame}

\begin{frame}{Extendendo o grafo de operações básicas: negativo}
\begin{figure}[ht!]
\centering

\scalebox{1.5}{
\begin{tikzpicture}[auto]

% operations =============================
\node[op] (times) {$*-1$};
\node[op, left=20pt of times] (a) {$a$};
\node[gradient, above=30pt of a] (da) {$\frac{\partial f}{\partial a}$};
\node[textonly, right=0.1pt of da] {$=-1$};
\node[textonly, right=0.1pt of times] {$=f(a) =-a$};

% edges
\path[tedge] (a) -- (times);

\end{tikzpicture}
} % scalebox
\end{figure}

\end{frame}

\begin{frame}{Extendendo o grafo de operações básicas: exponenciação}
\begin{figure}[ht!]
\centering

\scalebox{1.5}{
\begin{tikzpicture}[auto]

% operations =============================
\node[op] (times) {$exp$};
\node[op, left=20pt of times] (a) {$a$};
\node[gradient, above=30pt of a] (da) {$\frac{\partial f}{\partial a}$};
\node[textonly, right=0.1pt of da] {$=e^{a}$};
\node[textonly, right=0.1pt of times] {$=f(a) = e^{a}$};

% edges
\path[tedge] (a) -- (times);
\path[tedge] (a) -- (da);

\end{tikzpicture}
} % scalebox
\end{figure}

\end{frame}

\begin{frame}{Extendendo o grafo de operações básicas: logarítimo}
\begin{figure}[ht!]
\centering

\scalebox{1.5}{
\begin{tikzpicture}[auto]

% operations =============================
\node[op] (times) {$\log$};
\node[op, left=20pt of times] (a) {$a$};
\node[gradient, above=30pt of a] (da) {$\frac{\partial f}{\partial a}$};
\node[textonly, right=0.1pt of da] {$=\frac{1}{a}$};
\node[textonly, right=0.1pt of times] {$=f(a)=\log(a)$};

% edges
\path[tedge] (a) -- (times);
\path[tedge] (a) -- (da);

\end{tikzpicture}
} % scalebox
\end{figure}

\end{frame}

\begin{frame}{Extendendo o grafo de operações básicas: ao quadrado}
\begin{figure}[ht!]
\centering

\scalebox{1.5}{
\begin{tikzpicture}[auto]

% operations =============================
\node[op] (times) {$squ$};
\node[op, left=20pt of times] (a) {$a$};
\node[gradient, above=30pt of a] (da) {$\frac{\partial f}{\partial a}$};
\node[textonly, right=0.1pt of da] {$=2a$};
\node[textonly, right=0.1pt of times] {$=f(a)=a^2$};

% edges
\path[tedge] (a) -- (times);
\path[tedge] (a) -- (da);

\end{tikzpicture}
} % scalebox
\end{figure}

\end{frame}


\begin{frame}{Regra da cadeia}
\Large{
\begin{itemize}
\item $f:\mathbb{R} \rightarrow\mathbb{R}$, $g:\mathbb{R} \rightarrow\mathbb{R}$. 
\item $y = g(x)$
\item $u = f(g(x)) = f(y)$

\end{itemize}

\begin{figure}[ht!]
\centering

\scalebox{0.7}{
\begin{tikzpicture}[auto]

% operations =============================
\node[op] (x) {$x$};
\node[op, right=40pt of x] (y) {$y$};
\node[op, right=40pt of y] (u) {$u$};


% edges =============================
\path[tedge] (x) -- (y);
\path[tedge] (y) -- (u);
\end{tikzpicture}
} % scalebox
\end{figure}


\[
\frac{\partial u}{\partial x} = \frac{\partial u}{\partial y} \frac{\partial y}{\partial x} 
\]
}
\end{frame}

\begin{frame}{Aplicando a regra da cadeia}
\begin{figure}[ht!]
\centering

\scalebox{1.0}{
\begin{tikzpicture}[auto]

% operations =============================
\node[op] (nt) {$u_j$};
\node[op, above left=50pt of nt] (a) {$u_{j-1}$};
\node[op, below left=40pt of nt] (b) {$u_{j-2}$};
\node[op, right=40pt of nt] (ntp) {$u_{j+1}$};
\node[textonly, right=20pt of ntp] (ntpp) {$\dots$};
\node[op, right=40pt of ntpp] (nT) {$u_{n}$};
\node[gradient, above=15pt of nt] (dnt) {$\frac{\partial u_{j+1}}{\partial u_{j}}$};
\node[gradient2, above=15pt of ntp] (dntp) {$\frac{\partial u_{n}}{\partial u_{j+1}}$};
\node[gradient2, above=15pt of dnt] (ddnt) {$\frac{\partial u_{n}}{\partial u_{j}}$};
\node[gradient2, above=20pt of nT] (dLdL) {$\frac{\partial u_{n}}{\partial u_{n}}$};
\node[textonly, right=0.1pt of dLdL] {$=1$};

% edges =============================
\path[tedge] (a) -- (nt);
\path[tedge] (b) -- (nt);
\path[tedge] (nt) -- (ntp);
\path[tedge] (nt) -- (dnt);
\path[tedge] (ntp) -- (ntpp);
\path[tedge] (ntpp) -- (nT);
\path[tedge] (ntp) -- (dntp);
\path[tedge] (dnt) -- (ddnt);
\path[tedge] (dntp) -- (ddnt);

\end{tikzpicture}
} % scalebox
\end{figure}

\Large{
\begin{itemize}
\item $\frac{\partial u_{n}}{\partial u_{j}} = \frac{\partial u_{n}}{\partial u_{j+1}} \frac{\partial u_{j+1}}{\partial u_{j}}$
\end{itemize}
}

\end{frame}

\begin{frame}{Exemplo 1: regressão linear}
\Large{
\begin{align*}
J(\vect{w}) & = \frac{1}{N}\sum_{i=1}^{N}L(y_{i}, \hat{y}_{i})\\
            & = \frac{1}{N}\sum_{i=1}^{N}(\hat{y}_{i} - y_{i})^{2}\\
            & = \frac{1}{N}\sum_{i=1}^{N}(\vect{w}^\top\vect{x}_{i} - y_{i})^{2}\\
\end{align*}
}
\end{frame}

\begin{frame}{Simplificação}
\Large{
\begin{itemize}
\item $\vect{w} = \begin{bmatrix}w_{1}  \\ w_{2}\end{bmatrix}$

\vspace{0.8cm}

\item $\vect{x} = \begin{bmatrix}x_1 \\ x_2\end{bmatrix}$ 

\end{itemize}
}
\end{frame}



\begin{frame}{Grafo de $L(\hat{y}, y)$}
\begin{figure}[ht!]
\centering

\scalebox{0.8}{
\begin{tikzpicture}[auto]

% operations =============================

% input x and W
\node[op] (w1) {$w_{1}$};
\node[op, below=10pt of w1] (x1) {$x_{1}$};
\node[op, below=20pt of x1] (w2) {$w_{2}$};
\node[op, below=10pt of w2] (x2) {$x_{2}$};

% multiplication
\node[op, below right=1pt and 40pt of w1] (mult1) {$*$};
\node[op, below right=1pt and 40pt of w2] (mult2) {$*$};
% sum
\node[op, below right=25pt and 20pt of mult1] (sum1) {$+$};
\node[op, right=65pt of sum1] (sum2) {$+$};
\node[op, below=35pt of sum2] (minus) {$-1$};
\node[op, left=25pt of minus] (y) {$y$};
\node[op, right=45pt of sum2] (squ) {$squ$};


%edges
\path[tedge] (w1) -- (mult1);
\path[tedge] (x1) -- (mult1);
\path[tedge] (w2) -- (mult2);
\path[tedge] (x2) -- (mult2);

\path[tedge] (mult1) -- (sum1);
\path[tedge] (mult2) -- (sum1);
\path[tedge] (sum1) -- (sum2);
\path[tedge] (y) -- (minus);
\path[tedge] (minus) -- (sum2);
\path[tedge] (sum2) -- (squ);
\end{tikzpicture}
} % scalebox
\end{figure}

\end{frame}


\begin{frame}{Caminho de $w_1$}
\input{TikzFiles/lr_graph1}
\end{frame}

\begin{frame}{Derivade de $L$ em relação a $w_1$}
\begin{figure}[ht!]
\centering

\scalebox{0.7}{
\begin{tikzpicture}[auto]

% operations =============================

% input x and W
\node[op] (w1) {$w_{1}$};
\node[op, below=10pt of w1] (x1) {$x_{1}$};
\node[op, below=20pt of x1] (w2) {$w_{2}$};
\node[op, below=10pt of w2] (x2) {$x_{2}$};

% multiplication
\node[op, below right=1pt and 40pt of w1] (mult1) {$*$};
\node[op, below right=1pt and 40pt of w2] (mult2) {$*$};
% sum
\node[op, below right=25pt and 20pt of mult1] (sum1) {$+$};
\node[op, right=65pt of sum1] (sum2) {$+$};
\node[op, below=35pt of sum2] (minus) {$-1$};
\node[op, left=25pt of minus] (y) {$y$};
\node[op, right=45pt of sum2] (squ) {$squ$};



%gradients 1
\visible<3->{\node[gradient, above=10pt of sum2] (dsum2) {$2(\hat{y} - y)$};}
\visible<5->{\node[gradient, above=10pt of sum1] (dsum1) {$1$};}
\visible<7->{\node[gradient, above=10pt of mult1] (dmult1) {$1$};}
\visible<9->{\node[gradient, above=10pt of w1] (dw1) {$x_1$};}

%gradients 2
\visible<2->{\node[gradient2, above=35pt of squ] (dLdL) {$1$};}
\visible<4->{\node[gradient2, above right=35pt and 15pt of dsum2] (dLdLpp) {$2(\hat{y} - y)$};}
\visible<6->{\node[gradient2, above left=35pt and 15pt of dsum2] (dLdsum2) {$2(\hat{y} - y)$};}
\visible<8->{\node[gradient2, above left=10pt and 25pt of dLdsum2] (dLdmult1) {$2(\hat{y} - y)$};}
\visible<10->{\node[gradient2, left=25pt of dLdmult1] (dLdw1) {$2(\hat{y} - y)x_1$};}




%edges
\path[tedge] (w1) -- (mult1);
\path[tedge] (x1) -- (mult1);
\path[tedge] (w2) -- (mult2);
\path[tedge] (x2) -- (mult2);

\path[tedge] (mult1) -- (sum1);
\path[tedge] (mult2) -- (sum1);
\path[tedge] (sum1) -- (sum2);
\path[tedge] (y) -- (minus);
\path[tedge] (minus) -- (sum2);
\path[tedge] (sum2) -- (squ);

\visible<4->{\path[tedge] (dLdL) -- (dLdLpp);}
\visible<4->{\path[tedge] (dsum2) -- (dLdLpp);}
\visible<3->{\path[tedge] (sum2) -- (dsum2);}
\visible<6->{\path[tedge] (dLdLpp) -- (dLdsum2);}

% \visible<4->{\path[tedge] (dsum2) -- (dLdsum2);}
\visible<6->{\path[tedge] (dsum1) -- (dLdsum2);}
\visible<8->{\path[tedge] (dLdsum2) -- (dLdmult1);}
\visible<8->{\path[tedge] (dmult1) -- (dLdmult1);}
\visible<10->{\path[tedge] (dLdmult1) -- (dLdw1);}
\visible<10->{\path[tedge] (dw1) -- (dLdw1);}


\end{tikzpicture}
} % scalebox
\end{figure}




\end{frame}


\begin{frame}{Regra da cadeia para várias variáveis}
\Large{
\begin{itemize}
\item $z = f(x,y)$
\item $x = f_{1}(a)$. 
\item $y = f_{2}(a)$
\end{itemize}
\[
\frac{\partial z}{ \partial a} = \frac{\partial z}{\partial x} \frac{\partial x}{\partial a} + \frac{\partial z}{\partial y} \frac{\partial y}{\partial a} 
\]
}
\end{frame}


\begin{frame}{Exemplo}
\begin{figure}[ht!]
\centering

\scalebox{1.2}{
\begin{tikzpicture}[auto]

% operations =============================
\node[op] (div) {$div$};
\node[textonly, right=0.1pt of div] {$=f(x,y)=\frac{x}{y}=\frac{a*c}{a+b}$};
\node[op, above left=20pt of div] (mult) {$*$};
\node[op, below left=20pt of div] (plus) {$+$};
\node[op, below=20pt of plus] (b2) {$b$};
\node[op, above=20pt of mult] (b1) {$c$};
\node[op, left=55pt of div] (a) {$a$};
\node[textonly, above right=5pt and 5pt of div] (inv1) {$\frac{\partial f}{\partial x} \frac{\partial x}{\partial a}$};
\node[textonly, above left=2pt and 2pt of a] (inv2) {};
\node[textonly, below right=5pt and 5pt of div] (inv3) {$\frac{\partial f}{\partial y} \frac{\partial y}{\partial a}$};
\node[textonly, below left=2pt and 2pt of a] (inv4) {};


% edges
\path[tedge] (a) -- (mult);
\path[tedge] (b1) -- (mult);
\path[tedge] (a) -- (plus);
\path[tedge] (b2) -- (plus);
\path[tedge] (plus) -- (div);
\path[tedge] (mult) -- (div);
\path[tedge, nephritis!60, line width=1mm] (inv1) to [out=120,in=80] (inv2);
\path[tedge, nephritis!60, line width=1mm] (inv3) to [out=-120,in=-80] (inv4);



\end{tikzpicture}
} % scalebox
\end{figure}

\end{frame}

\begin{frame}{Exemplo 2: regressão logística}
\Large{
\begin{equation*}
\hat{\vect{y}} = softmax(\vect{W}\vect{x} + \vect{b})
\end{equation*}
\begin{equation*}
L(\vect{y},\hat{\vect{y}}) = CE(\vect{y},\hat{\vect{y}})
\end{equation*}


\begin{equation*}
L(\vect{y},\hat{\vect{y}}) = - \sum_{i}\vect{y}_{i} \log \left(\frac{exp(\sum_{k} \vect{W}_{i,k}\vect{x}_{k} + \vect{b}_{i})}{\sum_{j}exp(\sum_{k}\vect{W}_{j,k}\vect{x}_{k} + \vect{b}_{j})} \right)
\end{equation*}
}
\end{frame}

\begin{frame}{Simplificação}
\Large{
\begin{itemize}
\item $\begin{bmatrix}z_1\\z_2\end{bmatrix} = \begin{bmatrix}w_{11}  & w_{12}\\w_{21}  & w_{22}\end{bmatrix}* \begin{bmatrix}x_1\\x_2\end{bmatrix} + \begin{bmatrix}b_1\\b_2\end{bmatrix}$

\vspace{0.4cm}

\item $\begin{bmatrix}h_1\\h_2\end{bmatrix} = \begin{bmatrix}exp(z_1)\\exp(z_2)\end{bmatrix}$ 

\vspace{0.4cm}

\item $H = h_1 + h_2$ 

\vspace{0.4cm}

\item $\begin{bmatrix}\hat{y}_1\\\hat{y}_2\end{bmatrix} = \begin{bmatrix}\frac{h_1}{H}\\\frac{h_2}{H}\end{bmatrix}$ 
\end{itemize}
}
\end{frame}



\begin{frame}{Grafo de $L(\hat{\vect{y}}, \vect{y})$}
\input{TikzFiles/expanded_graph_0}
\end{frame}

\begin{frame}{Caminho de $b_1$: 1}
\input{TikzFiles/b1_path1}
\end{frame}

\begin{frame}{Caminho de $b_1$: 2}
\input{TikzFiles/b1_path2}
\end{frame}

\begin{frame}{Caminho de $b_1$: 3}
\input{TikzFiles/b1_path3}
\end{frame}

\begin{frame}{Derivada parcial de L com respeito a $b_1$: 1}
\input{TikzFiles/b1_path1_grad}
\end{frame}

\begin{frame}{Derivada parcial de L com respeito a $b_1$: 2}
\begin{figure}[ht!]
\centering

\scalebox{0.7}{
\begin{tikzpicture}[auto]

% operations =============================

% multiplication
\node[op] (z1) {$z_1$};
\node[op, above left=25pt and 20pt of z1] (b1) {$b_1$};

% exp
\node[op, right=25pt of z1] (exp1) {$h_1$};
\node[op, right=25pt of exp1] (H) {$H$};
\node[op, right=35pt of H] (div1) {$\hat{y}_1$};

% log
\node[op, right=25pt of div1] (log1) {$\log$};
\node[op, right=25pt of log1] (mult5) {$*$};
\node[op, below left=25pt and 10pt of mult5] (y1) {$y_1$};
\node[op, right=25pt of mult5] (sum6) {$+$};
\node[op, right=25pt of sum6] (minus1) {$*-1$};
\node[textonly, below left=55pt and 10pt of sum6] (dots) {{\LARGE$\dots$}};

%gradients 1
\node[gradient, above=10pt of sum6] (dsum6) {$-1$};
\node[gradient, above=10pt of mult5] (dmult5) {$1$};
\node[gradient, above=10pt of log1] (dlog1) {$y_1$};
\node[gradient, above=10pt of div1] (ddiv1) {$\frac{H}{h_1}$};
\node[gradient, above=10pt of H] (dH) {$-\frac{h_1}{H^2}$};
\node[gradient, above=10pt of exp1] (dexp1) {$1$};
\node[gradient, above=10pt of z1] (dz1) {$h_1$};
\node[gradient, above=10pt of b1] (db1) {$1$};

%gradients 2
\node[gradient2, above=10pt of minus1] (dLdL) {$1$};
\node[gradient2, above right =30pt and 10pt of dsum6] (dLdLpp) {$-1$};
\node[gradient2, above left=30pt and 10pt of dsum6] (dLdmult5) {$-1$};
\node[gradient2, left=25pt of dLdmult5] (dLlog1) {$-y_1$};
\node[gradient2, left=25pt of dLlog1] (dLdiv1) {$-y_1\frac{H}{h_1}$};
\node[gradient2, left=25pt of dLdiv1] (dLdH) {$\frac{y_1}{H}$};
\node[gradient2, left=25pt of dLdH] (dLdexp1) {$\frac{y_1}{H}$};
\node[gradient2, left=25pt of dLdexp1] (dLdz1) {$y_1\frac{h_1}{H}$};
\node[gradient2, above=35pt of db1] (dLdb1) {$y_1\frac{h_1}{H}$};

%edges
\path[tedge] (b1) -- (z1);
\path[tedge] (z1) -- (exp1);
\path[tedge] (exp1) -- (H);
\path[tedge] (H) -- (div1);

\path[tedge] (div1) -- (log1);
\path[tedge] (log1) -- (mult5);
\path[tedge] (y1) -- (mult5);
\path[tedge] (mult5) -- (sum6);
\path[tedge] (dots) -- (sum6);
\path[tedge] (sum6) -- (minus1);

%edges gradient 1
% \path[tedge] (b1) -- (db1);
% \path[tedge] (z1) -- (dz1);
% \path[tedge] (exp1) -- (dexp1);
% \path[tedge] (div1) -- (ddiv1);
% \path[tedge] (H) -- (dH);
% \path[tedge] (log1) -- (dlog1);
% \path[tedge] (mult5) -- (dmult5);
% \path[tedge] (sum6) -- (dsum6);

%edges gradient 2
\path[tedge] (dsum6) -- (dLdLpp);
\path[tedge] (dLdL) -- (dLdLpp);
\path[tedge] (dLdLpp) -- (dLdmult5);
\path[tedge] (dmult5) -- (dLdmult5);
\path[tedge] (dLdmult5) -- (dLlog1);
\path[tedge] (dlog1) -- (dLlog1);
\path[tedge] (ddiv1) -- (dLdiv1);
\path[tedge] (dLlog1) -- (dLdiv1);
\path[tedge] (dexp1) -- (dLdexp1);
\path[tedge] (dH) -- (dLdH);
\path[tedge] (dLdiv1) -- (dLdH);
\path[tedge] (dLdH) -- (dLdexp1);
\path[tedge] (dz1) -- (dLdz1);
\path[tedge] (dLdexp1) -- (dLdz1);
\path[tedge] (db1) -- (dLdb1);
\path[tedge] (dLdz1) -- (dLdb1);


\end{tikzpicture}
} % scalebox
\end{figure}

\end{frame}

\begin{frame}{Derivada parcial de L com respeito a $b_1$: 3}
\input{TikzFiles/b1_path3_grad}
\end{frame}

\begin{frame}{Derivada parcial de L com respeito a $b_1$}
\Large{
\begin{align*}
\frac{\partial L}{\partial b_1} &= -y_1 + y_1\frac{h_1}{H} + y_2\frac{h_1}{H}\\
\vspace{0.2cm}
\visible<2->{&= y_1\left(\frac{h_1}{H} -1\right) + y_2\left(\frac{h_1}{H} -0\right)\\}
\visible<3->{&= y_1\left(\hat{y}_1 -1\right) + y_2\left(\hat{y}_1 -0\right)\\}
\visible<4->{&= \hat{y}_1 - y_1 \;\;\;\; \text{(quando} \;\; y \;\; \text{é um vetor one-hot)}}
\end{align*}
}
\end{frame}

\begin{frame}{Exemplo}
\begin{figure}[ht!]
\centering

\scalebox{1.3}{
\begin{tikzpicture}[auto]

% operations =============================

% nodes
\node[textonly] (W) {$\begin{bmatrix}0.65  & 1.19\\0.69  & -0.92\end{bmatrix}$};
\node[textonly, right=40pt of W] (x) {$\begin{bmatrix}0.2\\0.7\end{bmatrix}$};
\node[textonly, below=30pt of W] (b) {$\begin{bmatrix}0\\0\end{bmatrix}$};
\node[textonly, below=30pt of x] (y) {$\begin{bmatrix}1\\0\end{bmatrix}$};
\node[textonly, left=1pt of W] (Wname) {$\vect{W}=$};
\node[textonly, left=1pt of x] (xname) {$\vect{x}=$};
\node[textonly, left=1pt of b] (bname) {$\vect{b}=$};
\node[textonly, left=1pt of y] (yname) {$\vect{y}=$};


\end{tikzpicture}
} % scalebox
\end{figure}

\end{frame}

\begin{frame}{Forward}
\input{TikzFiles/expanded_graph_0}
\end{frame}

\begin{frame}{Forward}
\input{TikzFiles/expanded_graph_1}
\end{frame}

\begin{frame}{Forward}
\input{TikzFiles/expanded_graph_2}
\end{frame}

\begin{frame}{Forward}
\begin{figure}[ht!]
\centering

\scalebox{0.6}{
\begin{tikzpicture}[auto]

% operations =============================

% input x and W
\node[op] (w11) {$w_{11}$};
\node[op, below=10pt of w11] (x1) {$x_{1}$};
\node[op, below=20pt of x1] (w12) {$w_{12}$};
\node[op, below=10pt of w12] (x2) {$x_{2}$};

\node[op, below=20pt of x2] (w21) {$w_{21}$};
\node[op, below=10pt of w21] (x11) {$x_{1}$};
\node[op, below=20pt of x11] (w22) {$w_{22}$};
\node[op, below=10pt of w22] (x22) {$x_{2}$};

% multiplication
\node[op, below right=1pt and 20pt of w11] (mult1) {$*$};
\node[op, below right=1pt and 20pt of w12] (mult2) {$*$};
\node[op, below right=1pt and 20pt of w21] (mult3) {$*$};
\node[op, below right=1pt and 20pt of w22] (mult4) {$*$};

% sum
\node[state, below right=25pt and 20pt of mult1] (sum1) {$0.96$};
\node[state, below right=25pt and 20pt of mult3] (sum2) {$-0.5$};
\node[op, right=25pt of sum1] (sum3) {$z_1$};
\node[op, right=25pt of sum2] (sum4) {$z_2$};
\node[op, above left=25pt and 10pt of sum3] (b1) {$b_1$};
\node[op, below left=25pt and 10pt of sum4] (b2) {$b_2$};

% exp
\node[op, right=25pt of sum3] (exp1) {$h_1$};
\node[op, right=25pt of sum4] (exp2) {$h_2$};
\node[op, below right=65pt and 15pt of exp1] (sum5) {$H$};
\node[op, right=35pt of exp1] (div1) {$\hat{y}_1$};
\node[op, right=35pt of exp2] (div2) {$\hat{y}_2$};

% log
\node[op, right=25pt of div1] (log1) {$\log$};
\node[op, right=25pt of div2] (log2) {$\log$};
\node[op, right=25pt of log1] (mult5) {$*$};
\node[op, right=25pt of log2] (mult6) {$*$};
\node[op, above left=25pt and 10pt of mult5] (y1) {$y_1$};
\node[op, below left=25pt and 10pt of mult6] (y2) {$y_2$};
\node[op, below right=65pt and 15pt of mult5] (sum6) {$+$};
\node[op, right=25pt of sum6] (minus1) {$*-1$};



%edges
\path[tedge] (w11) -- (mult1);
\path[tedge] (x1) -- (mult1);
\path[tedge] (w12) -- (mult2);
\path[tedge] (x2) -- (mult2);
\path[tedge] (w21) -- (mult3);
\path[tedge] (x11) -- (mult3);
\path[tedge] (w22) -- (mult4);
\path[tedge] (x22) -- (mult4);

\path[tedge] (mult1) -- (sum1);
\path[tedge] (mult2) -- (sum1);
\path[tedge] (mult3) -- (sum2);
\path[tedge] (mult4) -- (sum2);
\path[tedge] (sum1) -- (sum3);
\path[tedge] (b1) -- (sum3);
\path[tedge] (sum2) -- (sum4);
\path[tedge] (b2) -- (sum4);

\path[tedge] (sum3) -- (exp1);
\path[tedge] (sum4) -- (exp2);
\path[tedge] (exp1) -- (sum5);
\path[tedge] (exp2) -- (sum5);
\path[tedge] (exp1) -- (div1);
\path[tedge] (exp2) -- (div2);
\path[tedge] (sum5) -- (div1);
\path[tedge] (sum5) -- (div2);


\path[tedge] (div1) -- (log1);
\path[tedge] (div2) -- (log2);
\path[tedge] (log1) -- (mult5);
\path[tedge] (y1) -- (mult5);
\path[tedge] (log2) -- (mult6);
\path[tedge] (y2) -- (mult6);
\path[tedge] (mult5) -- (sum6);
\path[tedge] (mult6) -- (sum6);
\path[tedge] (sum6) -- (minus1);

\end{tikzpicture}
} % scalebox
\end{figure}

\end{frame}

\begin{frame}{Forward}
\input{TikzFiles/expanded_graph_4}
\end{frame}

\begin{frame}{Forward}
\input{TikzFiles/expanded_graph_5}
\end{frame}

\begin{frame}{Forward}
\input{TikzFiles/expanded_graph_6}
\end{frame}

\begin{frame}{Forward}
\input{TikzFiles/expanded_graph_7}
\end{frame}

\begin{frame}{Forward}
\input{TikzFiles/expanded_graph_8}
\end{frame}

\begin{frame}{Forward}
\input{TikzFiles/expanded_graph_9}
\end{frame}

\begin{frame}{Forward}
\input{TikzFiles/expanded_graph_10}
\end{frame}

\begin{frame}{Forward}
\input{TikzFiles/expanded_graph_11}
\end{frame}

\begin{frame}{Forward}
\input{TikzFiles/expanded_graph_12}
\end{frame}

\begin{frame}{Forward}
\input{TikzFiles/expanded_graph_13}
\end{frame}

\begin{frame}{Backward}
\input{TikzFiles/expanded_graph_14}
\end{frame}


\begin{frame}{Algoritmo de back-propagation (caso escalar)}
\begin{algorithm}[H]
\begin{algorithmic}[1]
\STATE \textbf{Require:} Computational graph $\mathcal{G} = (\{ u_1, \dots, u_n \}, \mathcal{E}_1, \mathcal{E}_2)$, where $u_n$ is a leaf node.
\STATE Initialize $grad\_table$, a data structure that will store the derivatives that have been computed (at the end $grad\_table[u_i] = \frac{\partial u_n}{\partial u_i}$).
\STATE $grad\_table[u_n] \leftarrow 1$
\FOR{$j=n-1$ down to $1$}
\STATE $grad\_table[u_j] \leftarrow \sum_{u_{i} \in S(u_{j})}grad\_table[u_i]\frac{\partial u_i}{\partial u_j}$
\ENDFOR
\RETURN $grad\_table$
\end{algorithmic}
\caption{Back-propagation (scalar case)}
\label{alg:seq}
\end{algorithm}
\end{frame}


\begin{frame}[allowframebreaks]{Referências}

  \bibliography{my_references}
  \bibliographystyle{abbrv}

\end{frame}




\end{document}